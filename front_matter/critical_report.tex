\documentclass[tocstyle=ref]{ees}

\shortname{M. Haydn}
\shorttitle{Litaniæ de V. S}

\begin{document}

\eesTitlePage

\eesCriticalReport{
  – &     & –     & Pages 1 and 2 are missing in \A1. Thus, the original title
                    (which most likely included the scoring) has been lost.
                    \A1 contains staves for two trb and fag,
                    while \B1 and the two authentic copies
                    (D-Mbs Mus.ms. 4135/1, RISM~455022548;
                    A-Sd A 1476, RISM~659002089)
                    contain staves or parts for three trb.
                    Thus, it is likely that a bass trb may be used
                    instead of fag. As customary, trb 1/2 and fag play
                    colla parte with A, T, and B, respectively,
                    in tutti sections, even if \A1
                    lacks explicit directives. \\
  \midrule
  1 & 1–24 & –    & Since these bars are not available in \A1,
                    they employ \B1 as principal source.
                    Missing grace notes were tacitly added. \\
    & 21ff & coro & double caps spelling of “DEus” (bars 21, 25,
                    29, and 34) according to \A1 \\
    % & 4   & vl    & 3rd \quarterNote\ in \B1: f′8–e′16–d′16 \\
    % & 7   & org   & last \quarterNote\ in \B1: \crotchetRest \\
  \midrule
  2 & 131 & S     & double caps spelling of “DEus” according to \A1 \\
  \midrule
  5 & 118 & trb 2 & 1st \eighthNote\ in \A1: c′16–c′16 \\
  \midrule
  8 & 41–44 & vla & The 3rd \quarterNote\ in bar 41 to 1st \quarterNote\
                    in bar 44 are written as if in tenor clef. \\
  % 9 & 16  & vla   & in \B1 duplicate of bar 15 \\
  %   & 23  & vl 1  & 1st \halfNote\ in \B1: g″8.–d″16–d″8.–g″16 \\
  %   & 24  & trb 1 & last \eighthNote\ in \B1: a′8 \\
  %   & 25  & trb 3 & 2nd \halfNote\ in \B1: g8–g8–g8–g8 \\
  %   & 29  & trb 1, A & 1st \halfNote\ in \B1: a′2 \\
  %   & 37  & A     & last \quarterNote\ in \B1: f′8–\quaverRest \\
  %   & 37  & T     & last \quarterNote\ in \B1: d′8–\quaverRest \\
  %   & 37  & B     & last \quarterNote\ in \B1: f8–\quaverRest \\
}

\eesToc{
\begin{movement}{kyrie}
  Kyrie eleison,
  Christe eleison,
  Kyrie eleison.
  Christe, audi nos,
  Christe, exaudi nos.
  Pater de coelis, Deus, miserere nobis.
  Fili Redemptor mundi, Deus, miserere nobis.
  Spiritus Sancte, Deus, miserere nobis.
  Sancta Trinitas, unus Deus, miserere nobis.
\end{movement}

\begin{movement}{panisvivus}
  Panis vivus, qui de coelo descendisti,
  Deus absconditus et salvator,
  frumentum electorum,
  vinum germinans virgines,
  panis pinguis et deliciæ regum,
  juge sacrificium,
  oblatio munda,
  agnus absque macula,
  mensa purissima, angelorum esca,
  manna absconditum,
  memoria mirabilium Dei,
  miserere nobis.
\end{movement}

\begin{movement}{panissuper}
  Panis supersubstantialis,
  verbum caro factum, habitans in nobis,
  hostia sancta, calix benedictionis,
  mysterium fidei,
  miserere nobis.
\end{movement}

\begin{movement}{praecelsum}
  Præcelsum et venerabile Sacramentum,
  sacrificium omnium sanctissimum,
  vere propitiatorium pro vivis et defunctis,
  coeleste antidotum,
  quo a peccatis præservamur,
  miserere nobis.
\end{movement}

\begin{movement}{stupendum}
  Stupendum super omnia miracula,
  Sacratissima Dominicæ passionis commemoratio,
  donum transcendens omnem plenitudinem,
  memoriale præcipuum divini amoris,
  divinæ affluentia largitatis,
  sacrosanctum et augustissimum mysterium,
  pharmacum immortalitatis,
  tremendum ac vivificum Sacramentum,
  panis omnipotentia verbi caro factus,
  incruentum sacrificium,
  cibus et conviva,
  miserere nobis.
\end{movement}

\begin{movement}{dulcissimum}
  Dulcissimum convivium,
  cui assistunt Angeli ministrantes,
  Sacramentum pietatis,
  vinculum charitatis,
  offerens et oblatio,
  Spiritualis dulcedo in proprio fonte degustata,
  refectio animarum sanctarum,
  miserere nobis.
\end{movement}

\begin{movement}{viaticum}
  Viaticum in Domino morientium.
\end{movement}

\begin{movement}{pignus}
  Pignus futuræ gloriæ,
  miserere nobis.
\end{movement}

\begin{movement}{agnusdei}
  Agnus Dei, qui tollis peccata mundi: Parce nobis Domine.
  Agnus Dei, qui tollis peccata mundi: Exaudi nos Domine.
  Agnus Dei, qui tollis peccata mundi: Miserere nobis.
\end{movement}
}

\eesScore

\end{document}
